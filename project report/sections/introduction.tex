\section{Introduction}
\label{sec:introduction}


\subsection{Technology stack}
The primary objective of this project was to develop a prototype implementation of a web-based polling application.
This application allows users to create and participate in polls efficiently through a user-friendly interface. 
By leveraging web development frameworks and tools, the project aims to provide a robust and scalable platform for conducting polls.

To achieve this we made use of the following technology stack:

\begin{itemize}
    \item \textbf{Backend Framework and Language}
    \begin{itemize}
        \item Spring Boot with Kotlin
    \end{itemize}

    \item \textbf{Persistence Layer}
    \begin{itemize}
        \item Spring JPA  with database options PostgreSQL and MongoDB
    \end{itemize}

    \item \textbf{Frontend Framework}
    \begin{itemize}
        \item SvelteKit with Vite and TailwindCSS
    \end{itemize}

    \item \textbf{Messaging}
    \begin{itemize}
        \item Spring messaging with RabbitMQ
    \end{itemize}

    \item \textbf{Containerization and Deployment}
    \begin{itemize}
        \item Docker
    \end{itemize}
\end{itemize}


\subsection{Results}
The results obtained through this project highlight the feasibility and effectiveness of the proposed architecture with a usable website for running polls.
The prototype demonstrates key functionalities of the polling application, including poll creation, participation, and result aggregation.
Kotlin, in particular, proved advantageous for backend development, offering concise syntax and seamless integration with Java-based frameworks, which streamlined development and reduced boilerplate code.
The use of SvelteKit on the frontend ensures a responsive and engaging user experience, while the combination of Spring Boot, JPA and RabbitMQ in the backend provides a scalable and reliable infrastructure.
In addition, the inclusion of both relational and non-relational database options offers flexibility to cater to diverse use cases. \\

\subsection{Organization of report}
The remainder of this report is organized as follows:
\begin{itemize}
    \item \textbf{Section~\ref{sec:design}: Discusses the design of the application, outlining the design choices and user interface considerations.}
    \item \textbf{Section~\ref{sec:technology}: Evaluates Kotlin as a programming language for this project.}
    \item \textbf{Section~\ref{sec:implementation}: Delves into the prototype implementation, including frontend and backend.}
    \item \textbf{Section~\ref{sec:conclusion}: Concludes the report with a summary of findings and potential areas for future work.}
    \item \textbf{Section 6: References}
\end{itemize}
